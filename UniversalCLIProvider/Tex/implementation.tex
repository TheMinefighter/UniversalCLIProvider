 Die Implementation ist in C\#7.3 entwickelt worden.
 Sie ist fast vollständig mit XML Kommentaren\footnote{Siehe https://docs.microsoft.com/en-us/dotnet/csharp/programming-guide/xmldoc/xml-documentation-comments} dokumentiert.
 \subsection{Aufbau der Implementation}\label{subsec:Architecture}
 Die Implementation ist in 3 namespaces, sowie den root-namespace geteilt.
 In diesem liegt die Klasse \inlinecode{InvalidCLIConfigurationException}, eine Au\ss nahme, die ausgelöst wird, wenn die Bibliothek eine falsche Verwendung von Funktionen von seiten des Programmierers festgestellt hat.
 Allein durch das eingeben von inkorrekten Befehlen in die bereitgestellte Kommandozeilenoberfläche kann diese nicht ausgelöst werden. Au\
 ss erdem liegt in diesem namespace die \inlinecode{InterpretingOptions}-Klasse, die einige an vielen Stellen verwendete Optionen definiert.
 \subsubsection{UniversalCLIProvider.Attributes}
 Im \inlinecode{UniversalCLIProvider.Attributes} sind alle Attribute, die zur Definition von Kommandozeilenfunktionen verwendet werden können zusammengefasst, wie etwa das \inlinecode{CmdContextAttribute}.
 Alle diese Attribute speichern eine Referenz zu dem Programmelement auf das das Attribut angwendet wurde, sowie z.T. weitere Informationen die sich aus diesem ergeben.
 Folgende Klassen sind in diesem namespace vorhanden:
 \begin{itemize}
\item Das \inlinecode{CmdActionAttribute} wird verwendet um eine als Kommandozeilen-Aktion ausführbare Methode zu Kennzeichnen.
\item Das \inlinecode{CmdParameterAttribute} wird verwendet um ein Parameter einer Kommandozeilen-Aktion von der Kommandozeilenoberfläche modifizierbar machen.
\item Das \inlinecode{CmdParameterAliasAttribute} wird verwendet um einem Parameter einer Kommandozeilen-Aktion eine Kurzform zu geben, welche dem Parametr ein vordefinierten Wert zuweist.
\item Das \inlinecode{CmdContextAttribute} wird verwendet um eine Klasse als Kontext zu definieren, der u.a. Aktionen grupiert.
\item Das \inlinecode{CmdDefaultActionAttribute} wird verwendet um innerhalb eines Kontextes eine Aktion zu definieren die ausgeführt wird, wenn keine andere Aktion ausgewählt wurde.
\item Das Enum\footnote{Ein Enum ist ein Datentyp der nur die in ihm definierten Zustände annehmen kann.} \inlinecode{ContextDefaultActionPreset}
stellt verschieden Voreinstellung bereit, die alternativ zum \inlinecode{CmdContextAttribute} verwendet werden können.
\item Das \inlinecode{CmdConfigurationProviderAttribute} markiert eine Eigenschaft die eine Refernz zu einem Konfigurations Objekt(siehe~\ref{ConfigurationManagement}) bereitstellt.
\item Das \inlinecode{CmdConfigurationNamespaceAttribute} markiert dass eine Klasse Teil einer durch die bereitgestelte Kommandozeilenoberfläche verwaltbaren Konfiguration ist
\item Das Interface \inlinecode{IConfigurationRoot} kann von einer Klasse implementiert werden, die den Ursprung einer Kommandozeilen-verwalteten Konfiguration ist.
\item Das \inlinecode{CmdConfigurationFieldAttribute}markiert dass ein Feld oder eine Eigenschaft Teil einer durch die bereitgesttelte Kommandozeilenoberfläche verwaltbaren Konfiguration ist
\item Das Flag-Enum\footnote{Ein Flag-Enum ist ein Datentyp der mehrer mögliche Zustände aufzählt, die miteinander, auf Basis von Binär-Logik durch ||(Oder) Operatoren kombinierbar sind.}
\inlinecode{CmdParameterUsage} definiert in welchen Formen ein Parameter angegeben werden kann.
 \end{itemize}
 \subsubsection{UniversalCLIProvider.Internals}
 Im \inlinecode{UniversalCLIProvider.Internals} sind viel Funktionen und Datentypen beherbergt, die nur intern (innerhalb der Bibliothek) verwendet werden,
 und die Basis für viele weitere Funktionen stellen, die darrauf Aufbauen.
 Folgende Klassen sind in diesem namespace vorhanden:
 \begin{itemize}
  \item Die \inlinecode{CommandlineMethods} Klasse enthält die essentiellsten Funktionen für Kommandozeilen-Oberflächen, wie z.B. die Funktion zum parsen von Werten.
  \item Die \inlinecode{HexArgumentEncoding} Klasse enthält Funktionen, die für die Hexadezimalkodierung (siehe~\ref{Hexadecimalencoding}) benötigt werden.
  \item Die \inlinecode{ContextDefaultAction} Klasse stellt ein einheitliches, internes Interface für die Funktionen des \inlinecode{ContextDefaultActionPreset} Enums und des \inlinecode{CmdDefaultActionAttribute}.
  \item Die \inlinecode{ConfigurationHelpers} Klasse stellt Funktionen, die für das Konfigurations-Management (\ref{ConfigurationManagement}) nötig sind, bereit.
  \item Die \inlinecode{HelpGenerators} Klasse besteht aus Methoden die Hilfe, z.B. bei falschen Eingaben, generiert.
  Im zweiten Teil\footnote{In C\# können Klassen auf mehrere Dateien aufgeteilt werden; dies wird dann durch das \inlinecode{partial} Schlüsselwort in der Klassendeklaration angezeigt.} 
  (\inlinecode{HelpGenerators.Configuration}) sind Hilfe Funktionen für das Konfigurations-Managemententhalten.
 \end{itemize}
 \subsubsection{UniversalCLIProvider.Interpreters}
 Aufbauend auf den vorherigen beiden namespaces wird im \inlinecode{UniversalCLIProvider.Interpreters} die Hauptaufgabe der Bibliothek geleistet: 
 Das eigentliche interpretieren von Kommandozeilenbefehlen.
 Folgende Klassen sind in diesem namespace vorhanden:
 \begin{itemize}
  \item Die \inlinecode{CommandlineOptionInterpreter} Klasse ist die Basis f\"ur jede Interpretation, und erfüllt Aufgaben, die nur auf oberster Ebene durchgeführt werden müssen, wie z.B. die Speicherung der Argumente.
  \item Die \inlinecode{BaseInterpreter} Klasse ist die abstrakte Basis Klasse, von der alle folgenden Klassen erben.
  Sie definiert u.a. dass jeder Interpreter einen \inlinecode{Parent} hat, der auch ein \inlinecode{BaseInterpreter} ist, au\ss er es handelt sich um die Spitze eines solches Stapels,
  dann ist die \inlinecode{Parent}-Eigenschaft nämlich \inlinecode{null}.
  Au\ss erdem enthält sie die \inlinecode{Interpret} Methode, welche die Interpretation auf Basis eines gegebenen Ursprungs-Kontext startet, die wichtigste Methode zur Verwendung der Bibliothek.
  \item Die \inlinecode{ContextInterpreter} Klasse interpretiert einen gegeben Kommandozeilenkontext und wird von der CommandlineOptionInterpreter aufgerufen.
  Sie lädt dazu nötige Metadaten über inhalte des Kontext, sofern dies noch nicht geschehen ist.
  \item Die \inlinecode{ActionInterpreter} Klasse interpretiert eine gegebene Kommandozeilenaktion zusammen mit ihren Parametern.
  \item Die \inlinecode{ConfigurationInterpreter} Klasse interpretiert Befehle zum Konfigurations-Management.
 \end{itemize}
 \subsection{Erläuterung der Implementation der Klasse ContextInterpreter}\label{subsec:ContextInterpreter}
 Die Klasse \inlinecode{ContextInterpreter} erfüllt Schl\"usselfunktionen in dieser Bibliothek.
 Sie erbt von der abstrakten \inlinecode{BaseInterpreter} Klasse folgende Inhalte:
 \begin{itemize}
  \item Die \inlinecode{Name} Eigenschaft, die jedem Interpreter einen Namen gibt.
  \item Die \inlinecode{Offset} Eigenschaft vom Datentyp \inlinecode{int} gibt den Index des aktuell interpretierten Arguments an.
  \item Die \inlinecode{TopInterpreter} Eigenschaft referenziert einen \inlinecode{CommandlineOptionInterpreter}
  \item Die \inlinecode{Parent} Eigens chaft gibt das nächst höhere Element des Interpretations-Stack
   \footnote{Der Begriff Interpretations-Stack wird im folgenden verwendet um den Aufbau der Implementation des Interpretations-Vorgang zu beschreiben, da Interpretations-Aufgaben u.U. die Erfüllung kleinerer Interpretations-Aufgaben vorraussetzten, diese liegen im Interpretations-Stack darunter.}
   an.%TODO DEfine
  \item Die \inlinecode{PathBottomUp} Eigenschaft gibt ein \inlinecode{IEnumerable<BaseInterpreter>} der alle Elemente des Interpretations-Stack von unten nach oben aufz\"ahlt, basierend auf den \inlinecode{Parent} der einzelnen Elemente.
  \item Die \inlinecode{Path} Eigenschaft gibt den Interpretations-Stack von oben nach zur\"uck, indem \inlinecode{PathBottomUp} r\"uckw\"arts verwendet wird.
  \item Die Überschreibung der \inlinecode{ToString}, die nun die \inlinecode{Path} Eigenschaft formattiert zur\"uckgibt.
  \item Die \inlinecode{IncreaseOffset} Methode, die die \inlinecode{Offset} um 1 erh\"oht dann zur\"uckgibt ob weitere Parameter bereitstehen.
  \item Die abstrakte \inlinecode{Interpret} Methode, die den gegebenen Kontext interpretiert und zur\"uckgibt ob die Interpretation war.
  \item Die \inlinecode{IsParameterEqual} Methode,die pr\"uft ob ein gegebener Parameter einem erwarteten Parameter entspricht.
  \item Die \inlinecode{Reset} Methode setzt den \inlinecode{Offset} wieder auf 0 (f\"ur interaktiven Modus n\"otig).
  \item Zwei Konstruktoren, einen um einen \inlinecode{BaseInterpreter} als oberstes Element eines Interpretations-Stack auf Basis eines \inlinecode{CommandlineOptionInterpreter} zu erzeugen, und einen um einen bestehenden Interpretationsstack um eine Ebene nach unten zu erweitern.
 \end{itemize}
 Die \inlinecode{ContextInterpreter} Klasse selbst hat zudem das \inlinecode{UnderlyingContextAttribute}, dass das \inlinecode{CmdContextAttribute} referenziert, das den zu interpretierenden Kontext repr\"asentiert.
 Die beiden Konstruktoren werden um Werte f\"ur dieses Feld erweitert:
 \begin{lstlisting}[language={[Sharp]C}, title=Konstruktoren der ContextInterpreter Klasse]
internal ContextInterpreter([NotNull] CommandlineOptionInterpreter top,
 [NotNull] CmdContextAttribute underlyingContextAttribute, int offset = 0) :
 base(top, offset) =>
 UnderlyingContextAttribute = underlyingContextAttribute;

internal ContextInterpreter([NotNull] BaseInterpreter parent,
  [NotNull] CmdContextAttribute attribute, int offset = 0) :
  base(parent, attribute.Name, offset) =>
  UnderlyingContextAttribute = attribute; \end{lstlisting}
 Zuletzte \"uberschreibt sie die abstrakte Interpret Methode:
 \begin{lstlisting}[title=\"Uberschreibung der Interpret Methode]
  internal override bool Interpret() => Interpret(out ContextInterpreter _);\end{lstlisting}
 \subsection{Demonstration der Funktionen anhand der Referenzverwendung}\label{subsec:demonstration}
 \subsubsection{Einfache Aktionsausführungen}
 \subsubsection{Verwendung von Kontexten}
 \subsubsection{Hexadezimalkodierung}\label{Hexadecimalencoding}
 \subsubsection{Hilfe Funktion}
 \subsubsection{Konfigurations-Management}\label{ConfigurationManagement}
 \subsubsection{Interaktive Kommandozeilenoberfäche}
 \subsection{Vergleich meiner Lösung mit bisherigen Lösungen}\label{subsec:Comparison}