Die Implementation ist in C\#7.3 entwickelt worden.
Sie ist fast vollständig mit XML Kommentaren\footnote{Siehe https://docs.microsoft.com/en-us/dotnet/csharp/programming-guide/xmldoc/xml-documentation-comments} dokumentiert.
\subsection{Aufbau der Implementation}\label{subsec:Architecture}
Die Implementation ist in 3 namespaces, sowie den root-namespace geteilt.
In diesem liegt die Klasse \inlinecode{InvalidCLIConfigurationException}, eine Ausnahme, die ausgelöst wird, wenn die Bibliothek eine falsche Verwendung von Funktionen von seiten des Programmierers festgestellt hat.
Allein durch das eingeben von inkorrekten Befehlen in die bereitgestellte Kommandozeilenoberfläche kann diese nicht ausgelöst werden.
Au\ss erdem liegt in diesem namespace die \inlinecode{InterpretingOptions}-Klasse, die einige an vielen Stellen verwendete Einstellungen definiert.
\subsubsection{UniversalCLIProvider.Attributes}
Im \inlinecode{UniversalCLIProvider.Attributes} sind alle Attribute, die zur Definition von Kommandozeilenfunktionen verwendet werden können zusammengefasst, wie etwa das \inlinecode{CmdContextAttribute}.
Alle diese Attribute speichern eine Referenz zu dem Programmelement auf das das Attribut angwendet wurde, sowie z.T. weitere Informationen die sich aus diesem ergeben.
Folgende Klassen sind in diesem namespace vorhanden:
\begin{outline}
 \1 Das \inlinecode{CmdActionAttribute} wird verwendet um eine als Kommandozeilen-Aktion ausführbare Methode zu Kennzeichnen.
 \1 Das \inlinecode{CmdParameterAttribute} wird verwendet um einen Parameter einer Kommandozeilen-Aktion von der Kommandozeilenoberfläche modifizierbar machen.
 \1 Das \inlinecode{CmdParameterAliasAttribute} wird verwendet um einem Parameter einer Kommandozeilen-Aktion eine Kurzform zu geben, welche dem Parametr ein vordefinierten Wert zuweist.
 \1 Das \inlinecode{CmdContextAttribute} wird verwendet um eine Klasse als Kontext zu definieren, der u.a. Aktionen grupiert.
 \1 Das \inlinecode{CmdDefaultActionAttribute} wird verwendet um innerhalb eines Kontextes eine Aktion zu definieren die ausgeführt wird, wenn keine andere Aktion ausgewählt wurde.
 \1 Das Enum\footnote{Ein Enum ist ein Datentyp der nur die in ihm definierten Zustände annehmen kann.} \inlinecode{ContextDefaultActionPreset}
 stellt verschieden Voreinstellung bereit, die alternativ zum \inlinecode{CmdContextAttribute} verwendet werden können.
 \1 Das \inlinecode{CmdConfigurationProviderAttribute} markiert eine Eigenschaft die eine Refernz zu einem Konfigurations-Objekt (siehe~\ref{ConfigurationManagement}) bereitstellt.
 \1 Das \inlinecode{CmdConfigurationNamespaceAttribute} markiert, dass eine Klasse Teil einer durch die bereitgestelte Kommandozeilenoberfläche verwaltbaren Konfiguration ist
 \1 Das Interface \inlinecode{IConfigurationRoot} kann von einer Klasse implementiert werden, die den Ursprung einer Kommandozeilen-verwalteten Konfiguration ist.
 \1 Das \inlinecode{CmdConfigurationFieldAttribute}markiert, dass ein Feld oder eine Eigenschaft Teil einer durch die bereitgesttelte Kommandozeilenoberfläche verwaltbaren Konfiguration ist
 \1 Das Flag-Enum\footnote{Ein Flag-Enum ist ein Datentyp der mehrer mögliche Zustände aufzählt, die miteinander, auf Basis von Binär-Logik durch ||(Oder) Operatoren kombinierbar sind.}
 \inlinecode{CmdParameterUsage} definiert in welchen Formen ein Parameter angegeben werden kann.
\end{outline}
\subsubsection{UniversalCLIProvider.Internals}
Im \inlinecode{UniversalCLIProvider.Internals} sind viel Funktionen und Datentypen beherbergt, die nur intern (innerhalb der Bibliothek) verwendet werden,
und die Basis für viele weitere Funktionen stellen, die darrauf Aufbauen.
Folgende Klassen sind in diesem namespace vorhanden:
\begin{outline}
 \1 Die \inlinecode{CommandlineMethods} Klasse enthält die essentiellsten Funktionen für Kommandozeilen-Oberflächen, wie z.B. die Funktion zum parsen von Werten.
 \1 Die \inlinecode{HexArgumentEncoding} Klasse enthält Funktionen, die für die Hexadezimalkodierung (siehe~\ref{Hexadecimalencoding}) benötigt werden.
 \1 Die \inlinecode{ContextDefaultAction} Klasse stellt ein einheitliches, internes Interface für die Funktionen des \inlinecode{ContextDefaultActionPreset} Enums und des \inlinecode{CmdDefaultActionAttribute}.
 \1 Die \inlinecode{ConfigurationHelpers} Klasse stellt Funktionen, die für das Konfigurations-Management (\ref{ConfigurationManagement}) nötig sind, bereit.
 \1 Die \inlinecode{HelpGenerators} Klasse besteht aus Methoden die Hilfe, z.B. bei falschen Eingaben, generiert.
 Im zweiten Teil\footnote{In C\# können Klassen auf mehrere Dateien aufgeteilt werden; dies wird dann durch das \inlinecode{partial} Schlüsselwort in der Klassendeklaration angezeigt.}
 (\inlinecode{HelpGenerators.Configuration}) sind Hilfe Funktionen für das Konfigurations-Managemententhalten.
\end{outline}
\subsubsection{UniversalCLIProvider.Interpreters}
Aufbauend auf den vorherigen beiden namespaces wird im \inlinecode{UniversalCLIProvider.Interpreters} die Hauptaufgabe der Bibliothek geleistet:
Das eigentliche interpretieren von Kommandozeilenbefehlen.
Folgende Klassen sind in diesem namespace vorhanden:
\begin{outline}
 \1 Die \inlinecode{CommandlineOptionInterpreter} Klasse ist die Basis für jede Interpretation, und erfüllt Aufgaben, die nur auf oberster Ebene durchgeführt werden müssen, wie z.B. die Speicherung der Argumente.
 \1 Die \inlinecode{BaseInterpreter} Klasse ist die abstrakte Basis Klasse, von der alle folgenden Klassen erben.
 Sie definiert u.a. dass jeder Interpreter einen \inlinecode{Parent} hat, der auch ein \inlinecode{BaseInterpreter} ist, au\ss er es handelt sich um die Spitze eines solches Stapels,
 dann ist die \inlinecode{Parent}-Eigenschaft nämlich \inlinecode{null}.
 Au\ss erdem enthält sie die \inlinecode{Interpret} Methode, welche die Interpretation auf Basis eines gegebenen Ursprungs-Kontext startet, die wichtigste Methode zur Verwendung der Bibliothek.
 \1 Die \inlinecode{ContextInterpreter} Klasse interpretiert einen gegeben Kommandozeilenkontext und wird von der CommandlineOptionInterpreter aufgerufen.
 Sie lädt dazu nötige Metadaten über inhalte des Kontext, sofern dies noch nicht geschehen ist.
 \1 Die \inlinecode{ActionInterpreter} Klasse interpretiert eine gegebene Kommandozeilenaktion zusammen mit ihren Parametern.
 \1 Die \inlinecode{ConfigurationInterpreter} Klasse interpretiert Befehle zum Konfigurations-Management.
\end{outline}
\subsection{Erläuterung der Implementation der Klasse ContextInterpreter}\label{subsec:ContextInterpreter}
Die Klasse \inlinecode{ContextInterpreter} erfüllt Schlüsselfunktionen in dieser Bibliothek.
Sie erbt von der abstrakten \inlinecode{BaseInterpreter} Klasse folgende Inhalte:
\begin{outline}
 \1 Die \inlinecode{Name} Eigenschaft, die jedem Interpreter einen Namen gibt.
 \1 Die \inlinecode{Offset} Eigenschaft vom Datentyp \inlinecode{int} gibt den Index des aktuell interpretierten Arguments an.
 \1 Die \inlinecode{TopInterpreter} Eigenschaft referenziert einen \inlinecode{CommandlineOptionInterpreter}
 \1 Die \inlinecode{Parent} Eigens chaft gibt das nächst höhere Element des Interpretations-Stack
 \footnote{Der Begriff Interpretations-Stack wird im folgenden verwendet um den Aufbau der Implementation des Interpretations-Vorgang zu beschreiben, da Interpretations-Aufgaben u.U. die Erfüllung kleinerer Interpretations-Aufgaben vorraussetzten, diese liegen im Interpretations-Stack darunter.}
 an.%TODO DEfine
 \1 Die \inlinecode{PathBottomUp} Eigenschaft gibt ein \inlinecode{IEnumerable<BaseInterpreter>} der alle Elemente des Interpretations-Stack von unten nach oben aufzählt, basierend auf den \inlinecode{Parent} der einzelnen Elemente.
 \1 Die \inlinecode{Path} Eigenschaft gibt den Interpretations-Stack von oben nach zurück, indem \inlinecode{PathBottomUp} rückwärts verwendet wird.
 \1 Die Überschreibung der \inlinecode{ToString}, die nun die \inlinecode{Path} Eigenschaft formattiert zurückgibt.
 \1 Die \inlinecode{IncreaseOffset} Methode, die en \inlinecode{Offset} um 1 erhöht dann zurückgibt ob weitere Parameter bereitstehen.
 \1 Die abstrakte \inlinecode{Interpret} Methode, die den gegebenen Kontext interpretiert und zurückgibt ob die Interpretation war.
 \1 Die \inlinecode{IsParameterEqual} Methode,die prüft ob ein gegebener Parameter einem erwarteten Parameter entspricht.
 \1 Die \inlinecode{Reset} Methode setzt den \inlinecode{Offset} wieder auf 0 (für interaktiven Modus nötig).
 \1 Zwei Konstruktoren, einen um einen \inlinecode{BaseInterpreter} als oberstes Element eines Interpretations-Stack auf Basis eines \inlinecode{CommandlineOptionInterpreter} zu erzeugen, und einen um einen bestehenden Interpretationsstack um eine Ebene nach unten zu erweitern.
\end{outline}
Die \inlinecode{ContextInterpreter} Klasse selbst hat zudem das \inlinecode{UnderlyingContextAttribute}, dass das \inlinecode{CmdContextAttribute} referenziert, das den zu interpretierenden Kontext repräsentiert.
Die beiden Konstruktoren werden um Werte für dieses Feld erweitert:
\begin{lstlisting}[language={[Sharp]C}, title=Konstruktoren der ContextInterpreter Klasse]
 internal ContextInterpreter([NotNull] CommandlineOptionInterpreter top,
   [NotNull] CmdContextAttribute underlyingContextAttribute, int offset = 0) :
   base(top, offset) =>
   UnderlyingContextAttribute = underlyingContextAttribute;

 internal ContextInterpreter([NotNull] BaseInterpreter parent,
   [NotNull] CmdContextAttribute attribute, int offset = 0) :
   base(parent, attribute.Name, offset) =>
   UnderlyingContextAttribute = attribute;
\end{lstlisting}
Die \inlinecode{InteractiveInterpreter} Methode wird zur interaktiven Interpretation verwendet.
Optional kann der Methode der Wahrheitswert \inlinecode{InterpretOn} übergeben werden, der spezifiziert,
ob nach dem aufrufen den interakiven shell der Nutzer noch weitere Argumente spezifiziert hat, welche dann als erstes Kommando der interaktiven Shell interprertiert werden würden.
In der Methode wird als erstes die \inlinecode{currentContextInterpreter} Variable vom Typ \inlinecode{ContextInterpreter} deklariert und mit einer \inlinecode{this} Referenz initalisiert.
Sie gibt an in welchem Kontext sich der die Interaktive Interpretation aktuell befindet.
Dann startet eine \inlinecode{while (true)}-Schleife.
In dieser werden zuerst die Argumente für den nächsten Befehl ausgelesen, dann wird dieser interpretiert.

Zuerst wird geprüft ob \inlinecode{InterpretOn} gesetzt wurde.
Wenn ja wird es deaktiviert und alle weiteren Parameter werden als Befehl verwendet:
\begin{lstlisting}[title=""]
 if (interpretOn) {
  interpretOn = false;
  TopInterpreter.Args=TopInterpreter.Args.Skip(Offset).ToArray();
 }
\end{lstlisting}
Sonst werden die Parameter aus der Kommandozeilenoberfläche gelesen:
\begin{lstlisting}[title=""]
 else {
  Console.Write(string.Join(TopInterpreter.Options.InteractiveSubPathSeparator, currentContextInterpreter.Path.Select(x => x.Name)) + ">");
  string readLine = Console.ReadLine();
  if (readLine is null) {
   continue;
  }
  TopInterpreter.Args =
   CommandlineMethods.ArgumentsFromString(readLine).ToArray();
  } \end{lstlisting}
Dann wird das gegebene Kommand interpretiert:
\begin{lstlisting}[title=""]
 currentContextInterpreter.Reset();
 if (currentContextInterpreter.Interpret(
   out ContextInterpreter tmpContextInterpreter, true)) {
  currentContextInterpreter = tmpContextInterpreter;
 }
\end{lstlisting}
Danach ist die Schleife und die Methode zu Ende.

Die \inlinecode{Interpret} Methode macht die eigentliche Interpretationsarbeit.
Sie nimmt als Parameter einen Wahrheitswer, ob die Interpretation von einer ineraktiven Kommandozeilenoberfläche kommt.
Zurückgegeben wird, ob die Interpretation erfolgreich war, und den neuen Interpretationskontext.
Zuerst wird getestet, ob die Interpretation interaktiv ist und das erste Argument \inlinecode{..} ist.
Wenn ja der wird nächst höhere Interpreter zurückgegeben, bzw. wenn der aktuelle schon der höchste ist das Programm verlassen:
\begin{lstlisting}[title=""]
if (interactive && TopInterpreter.Args[Offset] == "..") {
 if (Parent == null) {
  Environment.Exit(0);
 }
 else {
  var parentInterpreter = Parent as ContextInterpreter;
  parentInterpreter.Reset();
  newCtx = parentInterpreter;
  return true;
 }
}
\end{lstlisting}
Wenn keine weiteren Argumente angegeben werden wird die Standard-Aktion des Kontext ausgeführt.
\begin{lstlisting}[title=""]
UnderlyingContextAttribute.Load();
newCtx = null;
if (Offset >= TopInterpreter.Args.Length) {
 UnderlyingContextAttribute.DefaultAction.Interpret(this);
 return true;
}
\end{lstlisting}
Wenn weitere Argumente vorliegen wird zuerst nach Subkontexten gesucht:
\begin{lstlisting}[title=""]
string search = TopInterpreter.Args[Offset];
foreach (CmdContextAttribute cmdContextAttribute in 
  UnderlyingContextAttribute.SubCtx) {
 if (IsParameterEqual(cmdContextAttribute.Name, search, cmdContextAttribute.ShortForm, interactive)) {
  if (IncreaseOffset()) {
   if (interactive) {
    newCtx = new ContextInterpreter(this, cmdContextAttribute);
    return true;
   }
   else {
    return false;
   }
  }

  var subInterpreter = 
   new ContextInterpreter(this, cmdContextAttribute, Offset);
  if (interactive) {
   return subInterpreter.Interpret(out newCtx, true);
  }
  else {
   subInterpreter.Interpret();
  }

  return true;
 }
}
\end{lstlisting}
Wenn kein passender Subkontext gefund wurde wird nach passenden Aktionen gesucht:
\begin{lstlisting}[title=""]
foreach (CmdActionAttribute cmdActionAttribute in UnderlyingContextAttribute.CtxActions) {
 if (IsParameterEqual(cmdActionAttribute.Name, search, cmdActionAttribute.ShortForm, true)) {
  IncreaseOffset();
  var actionInterpreter = new ActionInterpreter(cmdActionAttribute, this, Offset);
  if (!actionInterpreter.Interpret()) {
  	HelpGenerators.PrintActionHelp(cmdActionAttribute, this);
  }
  newCtx = this;
  return true;
 }
}
\end{lstlisting}
Wenn auch das scheitert wird nach Kommandozeilenverwalteten Konfigurationen in diesem Kontext gesucht:
\begin{lstlisting}[title=""]
foreach (CmdConfigurationProviderAttribute provider in UnderlyingContextAttribute.CfgProviders) {
 if (IsParameterEqual(provider.Name, search, provider.ShortForm, true)) {
  var cfgInterpreter = new ConfigurationInterpreter(TopInterpreter, provider.Root, provider.UnderlyingPropertyOrField.GetValue(null),
  provider.UnderlyingPropertyOrField.ValueType.GetTypeInfo());
  return cfgInterpreter.Interpret();
 }
}
\end{lstlisting}
Wenn auch hier dabei nichts gefunden wird, eine Fehlermeldung und die Hilfe ausgegeben, und \inlinecode{false} zurückgegeben.

Zuletzt überschreibt sie die abstrakte Interpret Methode:
\begin{lstlisting}[title=""]
 internal override bool Interpret() => Interpret(out ContextInterpreter _);
\end{lstlisting}
Dann ist die Klasse zuende.
\subsection{Demonstration der Funktionen anhand der Referenzverwendung}\label{subsec:demonstration}
Um alle Funktionen zu demonstrieren habe ich eine Referenzverwendung der Bibliothek implementiert (Projekt \inlinecode{ReferenceUsage}).
Sie besteht aus der \inlinecode{Program} Klasse, die die Interpretationsfunktion der Bibliothek aufruft, der \inlinecode{CmdRootContext} Klasse, 
die den Ursprungskontext darstellt, sowie die \inlinecode{ReferenceConfig} Klasse, die eine typische Programmkonfiguration darstellen soll.

Alle folgenden Befehle können in einer normalen Kommandozeile ausgeführt werden.
Weitere Informationen dazu finden sich in Abschnitt~\ref{sec:Attachments}.
\subsubsection{Einfache Aktionsausführungen}
Ein Aktions Attribut bezieht sich immer auf eine Mthode.
Der einfachste Befehl wäre der eine Aktion ausführt ist \inlinecode{ReferenceUsage --BaseTest}.
Er führt die \inlinecode{BaseTest} Aktion aus.
Sie gibt \inlinecode{You have successfully entered BaseTest action.} aus.
Da die Kurzform \inlinecode{-b} definiert ist, kann auch \inlinecode{ReferenceUsage -b} genutzt werden.

Weitere Eigenschaften einer Aktion sind die Beschreibung, die lange Beschreibung und Beispiele, die alle ausschlie\ss lich zum generiernen der Hilfe verwendet werden.

Der nächste Befehl ist die \inlinecode{TestB} Aktion.
Sie benötigt zusätzlich noch das \inlinecode{Argumen-One} Argumentvom Typ \inlinecode{string}.
Eine einfache Ausführung ist \inlinecode{ReferenceUsage --TestB --Argument-One Test}.
Das gibt dann \inlinecode{You have used "Test" as argument.} aus.

Auch das Argument hat eine Kurzform (\inlinecode{-o}) die verwendet werden kann: \inlinecode{ReferenceUsage --TestB -o Test}.
Wenn das Argument weggelassen wird (\inlinecode{ReferenceUsage --TestB}), wird der definierte Standard-Wert für dieses verwendet:
\inlinecode{You have used "DefaultValue" as argument.}

Die Verwendung von Arrays wird durch \inlinecode{TestC} gezeigt.
Die Aktion hat den Parameter \inlinecode{--TestArg}, der vom Typ \inlinecode{string[]} ist.
Solche Parameter können auf zwei Wegen bereitgestellt werden:
\begin{outline}
 \1 Einfache Auflistung von Werten, z.B. \inlinecode{ReferenceUsage --TestC --TestArg Hallo Welt}, was in der Ausgabe \inlinecode{TestArg has the following elements: Hallo, Welt} resultiert.
 Wenn bei dieser Methode ein Element dem auch eine Parameter- oder Aliasdeklaration sein kan wird damit der Array beendet.
 \1 Alternativ kann ein JSON-Array verwendet werden: \inlinecode{ReferenceUsage --TestC --TestArg "[\\"Hallo\\",\\"Welt\\"]"}.
 Die Verwendng von Backslashes ist der Windows \inlinecode{cmd.exe} geschuldet, die keine Anführungszeichen ohne Backslash erlaubt,
 und auch mit Backslash nur wenn das Argument selbst in Anführungszeichen steht.
 Im Gegensatz zur vorherigen Methode ist es hier Möglich einen leeren Array (\inlinecode{[]}) zu übergeben.
 Au\ss erdem können bei dieser Methode auch ohne Probleme Elemente, die den Namen von weiteren Parametern entsprechen, verwenden.
 JSON Objekte können auch für das übergeben von Daten anderer Typen (ausnahme \inlinecode{string}) verwendet werden.
\end{outline} 

Des Weiteren können vor Parametern Aliasse deklariert werden.
Diese haben einen Namen mit dem sie aufgerufen werden können, sowie einen Wert, den sie repräsentieren.
So ist wird im eben verwenden Beispiel \inlinecode{["Hello","World"]} durch \inlinecode{--HelloW} repräsentiert.
Die Werte die durch einen Alias repräsentiert werden können sind beschränkt, da C\# Attribute nur bestimmte Typen von Werten unterstützen\footnote{Weitere Informationen sind unter https://docs.microsoft.com/en-us/dotnet/csharp/language-reference/language-specification/attributes\#attribute-parameter-types verfügbar.}.
Diese können auch Kurzformen (\inlinecode{-h}, sowie Beschreibungen für die Hilfe haben.

Um die Verwendung all dieser Möglichkeiten zu definieren gibt es die \inlinecode{usage} Eigenschaft vom Typen \inlinecode{CmdParameterUsage}.
Sie ist standartmä\ss ig auf \inlinecode{CmdParameterUsage.Default} gesetzt, was darin resultiert, dass die Bibliothek entscheidet, welche Option sinnvoll ist.
Wen man diese Einstellung selber bestimmen möchte kann man jede mögliche Kombination aus den folgenden drei Punkten angeben:
\begin{outline}
 \1 Unterstützung von Roh-Werten: \inlinecode{ReferenceUsage --TestC --TestArg "[\\"Hello\\",\\"World\\"]"}
 \1 Unterstützung von Aliassen mit vorheriger Parameter Deklaration: \inlinecode{ReferenceUsage --TestC --TestArg --HelloW}
 \1 Unterstützung von Aliassen ohne vorheriger Parameter Deklaration: \inlinecode{ReferenceUsage --TestC --HelloW}
\end{outline}

Parameter besitzen des Weiteren auch eine normale und eine lange Beschreibung.
\subsubsection{Verwendung von Kontexten}
Kontexte bieten die Möglichkiet Aktionen zu ordnen.
Ein Kontext Attribut bezieht sich immer auf eine Klasse.
Wenn dieser Kontext nicht der Ursprungskontext einer Kommandozeilenoberfläche ist, 
muss die Klasse auf die sich das Attribut bezieht ein geschachtelter Typ in einer Klasse, die wiederum  auch ein \inlinecode{CmdContextAttribute} besitzt, sein.
Nur in einer diesen Bedingungen entsprechenden Klasse können Aktionen, Kommandozeilenverwaltete Konfiguratioen und Kontext Standard Aktionen deklariert werden.

Wenn nun eine Aktion aufgerufen werden soll, die sich nicht im Ursprungskontext befindet,
muss der gesamte Pfad bis zum gewünschten Kontext, durch die Angabe der Namen der Kontexte in der Kommandozeilenoberfläche übergeben werden.
So befindet sich zum Beispiel die \inlinecode{TestZ} Aktion im \inlinecode{XZY} Kontext,
um die Aktion muss also der Befehl \inlinecode{ReferenceUsage --XZY --TestZ} verwendet werden;
um die \inlinecode{TestY} Aktion im \inlinecode{ABC} Kontext, der sich im \inlinecode{XYZ} Kontext befindet,
aufzurufen muss man \inlinecode{ReferenceUsage --XZY --ABC --TestZ} nutzen.

Kontexte können eine \inlinecode{ContextDefaultAction} definieren, die ausgeführt wird, wenn der Kontext angegeben wird, aber keine weiteren Argumente angegeben werden.
Diese kann auf zwei wegen definiert werden:
\begin{outline}
 \1 Im \inlinecode{CmdContextAttribute} kann ein \inlinecode{ContextDefaultActionPreset} angegeben werden, um eine der Vorlagen zu verwenden, 
 so ist zum Beispiel für den \inlinecode{XYZ} Kontext die \inlinecode{Exit} Vorlage aktiv,
 sodass sich die Kommandozeilenoberfläche bei der Eingabe von \inlinecode{ReferenceUsage --XZY} beendet.
 Es gibt folgende Vorlagen:
 \2 \inlinecode{Exit} beendet die Kommandozeilenoberfläche
 \2 \inlinecode{Help} zeigt die Hilfe an.
 \2 \inlinecode{Interactive} startet eine Interaktive Kommandozeilenoberfläche in dem Kontext 
 \1 Die Deklaration einer Methode im Kontext, die mit dem \inlinecode{CmdDefaultActionAttribute} gekennzeichnet ist,
 welches zusätzlich Parameter für die Methode enthalten kann.
 Wenn die Methode ein Parameter mehr erwartet als bereitgestellt wurde, wird als letzter Parameter der aktuelle \inlinecode{CmdContextInterpreter} übergeben.
 
 Der \inlinecode{CustomizedDefaultAction} Kontext besitzt eine Methode \inlinecode{TheDefaultAction}, die eine Zahl und einen \inlinecode{CmdContextInterpreter} als Parameter erwartet.
 Diese ist mit einem \inlinecode{CmdDefaultAction} Attribut gekennzeichnet, das mit der Zahl 7 parametriert ist.
 Wenn nun der Befehl \inlinecode{ReferenceUsage --CustomizedDefaultAction} ausgeführt wird, wird die \inlinecode{TheDefaultAction} Methode ausgeführt, die die Zahl, sowie den Pfad des Interpreters ausgibt:
 \inlinecode{The CmdDefaultActionAttribute supplied the number 7, the interpreter is currently located in the following path: ReferenceUsage>CustomizedDefaultAction}
\end{outline}
 Wenn beide Methoden verwendet werden, wird die Aktion des \inlinecode{CustomizedDefaultAction} ausgeführt, wenn keine spezifiziert wurde wird die Hilfe für den Kontext ausgegeben.

\subsubsection{Konfigurations-Management}\label{ConfigurationManagement}
Die Bibliothek bietet die Möglichkeit Konfigurationsdateien von der Kommandozeilenoberfläche aus zu verwalten.
Die Konfiguration wird dabei durch ein Objekt zur Laufzeit des Programms gespeichert, dass oft Referenzen zu anderen Konfigurationsteilen oder Mengen solcher enthält.
Eine Referenz zu so einem Konfigurationsobjekt muss in einer Eigenschaft in einem Kontext gespeichert werden.
Diese Eigenschaft muss mit einem \inlinecode{CmdConfigurationProvider} Attribut gekennzeichnet sein, das mit dem Namen parametriert werden muss, mit dem später die Konfiguration aufgerufen werden soll.
Die Klasse des Konfigurationsobjekts muss mit einem \inlinecode{CmdConfigurationNamespace} Attribut gekennzeichnet sein.
Werte darin, die Konfigurierbar sein sollen, müssen mit einem \inlinecode{CmdConfigurationFieldAttribute} gekennzeichnet werden.

Aktuell gibt es drei Operatoren für Konfigurationswerte:
\begin{outline}
 \1 Der \inlinecode{help} Operator gibt die Hilfe für das Feld aus.
 \1 Der \inlinecode{get} Operator gibt den Wert des Feldes aus.
 \1 Der \inlinecode{set} Operator setzt den Wet des Feldes auf den nachfolgenden Wert.
 Wenn die Konfigurationsklasse das \inlinecode{IConfigurationRoot} Interface implementiert, wird hierbei die \inlinecode{Save} Methode aufgerufen um die \"Anderungen zu speichern.
 \end{outline}

In der Referenzverwendung gibt es eine Konfiguration unter der \inlinecode{--cfg} Option, die u. a. ein Feld namens \inlinecode{IntA} des Typ \inlinecode{int} besitzt, das standardmä\ss den Wert 7 besitzt.
Um diesen Wert nun zu lesen muss der Befehl \inlinecode{ReferenceUsage --cfg IntA get} ausgeführt werden.
Um den Wert des Feld auf 5 zu setzten muss \inlinecode{ReferenceUsage --cfg IntA set 5} ausgeführt werden.

Arrays werden auch unterstützt\footnote{Eine Ausnahme ist der Typ \inlinecode{object[]}}.
Das Feld \inlinecode{ManyStrings} ist vom Typ \inlinecode{string[]}
Es können alle Werte des Arrays abgefragt werden (\inlinecode{ReferenceUsage --cfg ManyStrings get})
oder auch einzelne Elemente (\inlinecode{ReferenceUsage --cfg ManyStrings[1] get}) aber auch gesetzt werden (\inlinecode{ReferenceUsage --cfg ManyStrings[1] set ANewString}).
Es kann aber auch der gesamte Array ersetzt werden (\inlinecode{ReferenceUsage --cfg ManyStrings set "[\\"String1\\",\\"String2\\",\\"String3\\"]"}).

\subsubsection{Hilfs Funktion}
\subsubsection{Hexadezimalkodierung}\label{Hexadecimalencoding}
Au\ss erdem stellt sie Bibliothek die Funktion bereit, alle Programm-Argumente in ihrer Hexadezimalrepräsentation zu übergeben.
Die Hexadezimal Repräsentation  basiert dabei auf einer beliebigen Zeichenkodierung, jedoch wird im standardmä\ss ig UTF-8 (gemä\ss ~ISO/IEC 10646:2017) verwendet.

Die Kodierung besteht aus folgenden Teilen:
\begin{outline}
 \1 4 (Hexadezimal) Ziffern für die Zeichensatztabelle, die verwendet wurde um die Argumente zu kodieren
 \1 Für jedes Argument:
  \2 1 Ziffer die die Länge der Längenangabe angibt;
  muss zwischen 1 und 8 liegen 
  \2 1-8 Ziffern die die Länge des Arguments angeben
  \2 Das kodierte Argument, wobei zwischen 2 und 8 Ziffern pro Buchstabe verwendet werden
\end{outline}

Vor der Angabe solcher kodierten Argumente muss ein Argument stehen, dass dies kennzeichnet.
Dieses ist in den Einstellungen zur Interpretation als \inlinecode{HexOption} definiert und hat standardmä\ss ig den Wert \inlinecode{Master:Hex}.
So ist zum Beispiel der Befehl \inlinecode{ReferenceUsage --TestC --HelloW} äquivalent zum Befehl \inlinecode{ReferenceUsage --Master:Hex fde9172d2d5465737443182d2d48656c6c6f57}.

Eine Implemtation einer Funktion die aus einem \inlinecode{string[]} eine solche Hexadezimal-Zeichenfolge generiert ist unter dem Namen \inlinecode{ToHexArgumentString} in der \inlinecode{HexArgumentEncoding} zu finden.
Sie ist mit minimalen Modifikationen auch in vielen anderen Sprachen, wie Java, möglich.

Dieses Feature erlaubt es auch Sonderzeichen, wie Emojis, problemfrei zu übertragen was in normalen Kommandozeilenoberflächen, mit wenigen au\ss ahmen nicht möglich ist.
Au\ss erdem sin damit alle Probleme um das Verwenden von Backslashes gelöst.
Hier stellt sich natürlich die Frage weswegen ich nicht JSON hierfür verwendet habe.
Die Problematik liegt darin erstens immer Backslashes gesetzt werden müssen und, dass bestimmte Sonderzeichenfolgen in JSON nicht kodiert werden können, wie auch in der Spezifikation bemerkt~\cite{JSONSpec}.
Da sowieso ein Kodierungsschritt nötig ist und diese Funktion in jemandem Fall nur für automatisierte Szenarien zur Kommunikation zwischen Programmen sinnvoll ist,
ist die Verwendung eines etwas komplizierteren, proprietärem  Kodierungsprozess gegenüber fehlender Zuverlässigkeit und höherem Aufwand meiner Meinung nach gerechtfertigt.
\subsubsection{Interaktive Kommandozeilenoberfäche}

\subsection{Vergleich meiner Lösung mit bisherigen Lösungen}\label{subsec:Comparison}