Das Projekt ist, wie oben beschrieben schon jetzt sehr umfangreich, es gibt aber noch grö\ss eres Verbesserungspotential, welches ich in diesem Kapitel erläutern möchte.
\subsection{Ausbau der Kernfunktionen}\label{subsec:MoreFunctions}
Ich möchte in der Zukunft die Kernfunktionen der Bibliothek erweitern.
Zuerst möchte ich zum Konfigurationsmanagement zwei Operatoren hinzufügen: \inlinecode{Add} und \inlinecode{RemoveAt}, 
die auf Werte die das \inlinecode{ICollection} angewendet werden kann, um es zu ermöglichen Listen Elemente hinzuzufügen und entfernen zu können.
Dann überlege ich noch eine Funktion einzubauen, die es erlauben würde, Parameter zu deklarieren, die auf alle Werte in einem Kontext anwendbar sind.
\subsection{Übertragbarkeit auf andere Programmiersprachen}\label{subsec:PortabilityToOtherLangs}
Das Projekt ist in gro\ss en fast Vollständig in C\# ähnliche Sprache, wie Java, übertragbar, da fast alle modernen Sprachen ein Konzept von Annotationen oder Attributen implementieren.
Jedoch implementieren nicht alle Sprache ein System von Reflections, wie es für diese Bibliothek nötig ist, wie z.B. Rust.
\subsection{Geschwindigkeitsverbesserungen durch Speicherung von Reflections}\label{subsec:StoringReflections}
Die grö\ss te Verbesserungsmöglichkeit der Bibliothek liegt darin, die Geschwindigkeit zu verbessern,
da das interpretieren einer einfachen Aktion, einer Messung mit dotTrace zufolge, 452ms dauert.
410ms davon werden dafür benötigt die Reflections von Klassen abzurufen, eine Problem, das nicht nur in C\#, sondern auch z.B. in Java existiert.
Eine Lösung wäre es, die Reflections im vorraus zu speichern.
Dafür wäre zuerst ein Software Development Kit (SDK) nötig, dass all nötigen Reflections erfasst und abspeichert.
Des Weiteren wären geringe \"Anderungen in der Bibliothek nötig um externe Daten zu verwenden.
Dabei stellt sich die Frage, wie man diese Daten speichert:
\begin{outline}
 \1 Die Generierung einer C\# Klasse die in eine eigene Bibliothek kompiliert wird, dies würde jedoch die Einrichtung eines eigenen Projekts erfordern.
 \1 Die Generierung einer Bibliothek, die als Teil des Nutzer-Programms kompiliert wird, dies würde aber erfordern, dass das Nutzer Programm zwei Mal kompiliert wird.
 \1 Die Verwendung einer eingebunden z.B. JSON Datei, die zur Laufzeit der Bibliothek ausgewertet wird, dies könnte aber etwas langsamer als die anderen beiden Optionen sein.
\end{outline}
Eine endgültige Evaluation dieser Optionen steht noch aus.
Zuletzt wäre eine Integration in die C\# Build Pipeline wünschenswert, sowie Integrationen in dritt Werkzeuge (z.B. das Build-Pipeline
 Projekt\footnote{Weitere Informationen sind unter https://github.com/KBuroz/Build-Pipeline verfügbar}) und IDEs.
\subsection{Implementation einer Funktion für Grafische Kommandozeilenbenutzeroberfächen}
Eine weitere Verbesserugsmöglichkeit wäre die Implementation einer Funktion, die Grafische Kommandozeilenbenutzeroberfächen (GCLUI) generiert.
Diese würde die gleiche, auf Kontexten basierende, Ordungsstruktur, wie die normale Kommandozeilenbenutzeroberfächen verwenden.
Interessant wäre es zur Werte-\"Ubergabe einen grafischen Editor für Objekt Strukturen zu implementieren.
GCLUIs würden ein weiteres Alleinstellungsmerkmal der Bibliothek darstellen.
\subsection{Implementation einer eigenen Auto-Vervollständigung}\label{subsec:Autocomplete}
Die Implementation einer Auto-Vervollständigung würde der Bibliothek ein weiteres Schlüsselmerkmal gegenüber anderen C\# Bibliotheken geben.
Wenn dieses Feature aber auch in der Praxis praktisch sein soll, wäre es wichtig weitere Informationen über übergebenen Daten haben.
So wäre zum Beispiel ein Attribut für Parameter sinnvoll, das es ermöglicht zu definieren, das der Parameter ein Pfad eines Ordners, 
der existiert, ist, so dass die Auto-Vervollständigung entsprechende Werte vorschlagen kann.
Nur mit solchen Informationen kann eine Autovervollständigung im Alltäglichen arbeiten mit der Software einen Vorteil schaffen.
Im Folgenden möchte ich auf verschieden Umgebungen für automatische Vervollständigungen eingehen.
\subsubsection{Auto-Vervollständigung für die Interaktive Kommandozeilenoberfläche}
Eie Funktion zur automatischen Vervollständigung von Befehlen in der Interaktive Kommandozeilenoberfläche, würde die Verwendung dieser Potentiell deutlich attraktiver machen, 
da in einer interaktiven Kommandozeilenoberfäche es möglich ist, eine sehr umfangreiche Vervollständigung zu bieten.
Dabei gibt es viele unterschiedliche  Ausführungen, wie z.B. die der o.g. Python-Nubia Bibliothek, alle mit ihren Vor- und Nachteilen.
Eine umfangreiche Evaluation dieser wird der Implementation mindestens einer vorrausgehen. 
\subsubsection{Auto-Vervollständigung für Linux Kommandozeilen}
In vielen Linux Umgebungen ist es möglich das Verhalten zur Auto-Vervollständigung in der Systemeigenen Shell anzupassen.
Dafür muss eine Bash Datei mit vorgegebenen Funktionen vorhanden sein~\cite{BashAutoComplete}.
In der Zukunft wäre es denkbar, in diese Bibliothek eine Funktion zu implementieren, die es ermöglicht diese Dateien auf Basis der definierten Aktion,
Kontexte und anderer Eigenschaften vollautomatisch zu generieren.
\subsubsection{Auto-Vervollständigung für die Windows Powershell}
Die Windows Powershell hat eine Auto-Vervollständigungs Funktion, aber nur für Powershell Cmdlets.
Cmdlets bieten selber eine eigene einfache Kommandozeilenoberfläche, diese wird dann automatisch vervollständigt und dessen Eingaben werden automatisch in Objekte verarbeitet~\cite{CustomCmdlet}.
Daher wäre es dann nötig eine Funktion zu entwickeln, die für ein C\# Programm einen Powerdhell Cmdlet generiert, dass die gleichen möglichen Befehle hat, 
das dann in der Powershell automatisch Vervollständigt werden kann.
  Ich gehe davon aus das dieser Prozess sehr komplex wäre, und bin mir nicht sicher ob es möglich ist in dieser Umgebung, zum Beispiel für Konfigurationspfade, eine Vervollständigung bereitzustellen.
\subsection{Verbesserung der Unit-Test Abdeckung}\label{subsec:MoreUnitTests}
Die Bibliothek besitzt einige Unit-Tests im Projekt \inlinecode{UnitTests}, diese decken aber, einem Test mit dem Programm dotCover zufolge, 10\% der Befehle ab.
Es wäre bei einer Veröffentlichung der Bibliothek wünschenswert mindestens 95\% der Befehle automatisch testen zu können, damit eine Zuverlässige Weiterentwicklung möglich ist, 
da es sonst dazu kommen könnte das fehlerhafte Software Veröffentlicht wird.