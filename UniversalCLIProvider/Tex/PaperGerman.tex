
\documentclass[a4paper,12pt,titlepage,ngerman]{article}
\usepackage{babel}
\usepackage[T1]{fontenc}
\usepackage[utf8x]{inputenc}
\usepackage[a4paper,margin=2.5cm]{geometry}


% Kopf- und Fußzeilen
\usepackage{scrlayer-scrpage}
\setkomafont{pageheadfoot}{\textrm}
\ifoot{Tobias Brohl}
\cfoot{\thepage}
\ohead{}
% Für mathematische Befehle und Symbole
\usepackage{amsmath}
\usepackage{amssymb}

% Für Bilder
\usepackage{graphicx}

% Für Algorithmen
\usepackage{algpseudocode}

% Für Quelltext
\usepackage{listings}
\usepackage{color}
\definecolor{mygreen}{rgb}{0,0.6,0}
\definecolor{lightgray}{gray}{0.9}
\definecolor{mygray}{rgb}{0.5,0.5,0.5}
\definecolor{mymauve}{rgb}{0.58,0,0.82}
\lstset{
keywordstyle=\color{blue},commentstyle=\color{mygreen},
stringstyle=\color{mymauve},rulecolor=\color{black},
basicstyle=\footnotesize\ttfamily,numberstyle=\tiny\color{mygray},
captionpos=b, % sets the caption-position to bottom
keepspaces=true, % keeps spaces in text
numbers=left, numbersep=5pt,showstringspaces=true,
showtabs=false, stepnumber=2, tabsize=2, title=\lstname,
numbers=none,language={[Sharp]C},breaklines=true,frame=single,  literate={ö}{{ö}}1
{ä}{{ä}}1
{ü}{{ü}}1
{ß}{{\ss}}1
}

% Diese beiden Pakete müssen als letztes geladen werden
%\usepackage{hyperref} % Anklickbare Links im Dokument
\usepackage{cleveref}

% Daten für die Titelseite
\title{Entwicklung einer Bibliothek zum universellen bereitstellen von Kommandozeilenoberfächen}
\author{Tobias Brohl}
\subtitle{}
\date{\today \linebreak \linebreak Facharbeit im Fach Informatik (If GK 1)}

\parindent=0pt
\newcommand{\inlinecode}[2]{\colorbox{lightgray}{\lstinline[language=#1]$#2$}}
\begin{document}
 \begin{sloppypar}
  \maketitle
  \setcounter{tocdepth}{5}
  \tableofcontents
  \pagebreak
  \section{Einleitung \& Motivation}\label{sec:Intro}
  \subsection{Aktueller Stand der Technik}\label{subsec:CurrentState}
  \section{Beispiele}\label{sec:Content}
  \subsection{Aufbau der Implementation}\label{subsec:Architecture}
  \subsection{Demonstration der Funktionen anhand der Referenzverwendung}\label{subsec:demonstration-der-bibliothek-durch-die-referenzverwendung}
  \subsection{Vergleich meiner Lösung mit bisherigen Lösungen}\label{subsec:Comparison}
  \section{Ausblick}\label{sec:Future}
  \subsection{Ausbau des Funktionsumfangs}\label{subsec:MoreFunctions}
  \subsection{Geschwindigkeitsverbesserungen durch Speicherung von Reflections}\label{subsec:StoringReflections}
  \subsection{Verbesserung der Unit test Abdeckung}\label{subsec:MoreUnitTests}
  \section{Anmerkungen}\label{sec:AdditionalNotes}
  \subsection{Umgebungsvorraussetzuungen zum nutzen der Bibliothek}\label{subsec:SystemRequirements}
  \subsection{Verwendung von Bibliotheken}\label{subsec:UsageOfLibraries}
  \section{Fazit}\label{sec:Conclusion}
  \section{Anhang}\label{sec:anhang}
  \section{Literaturverzeichnis}\label{sec:Literature}
  \section{Selbstständigkeitserklärung}\label{sec:IDidThisMyself}
  Hiermit erkläre ich, dass ich die vorliegende Facharbeit selbstständig und ohne fremde Hilfe angefertigt und nur die im Literaturverzeichnis 
  aufgeführten Hilfen und Quellen benutzt habe.
  Insbesondere versichere ich, dass ich alle wörtlichen
  und sinngemä\ss en Übernahmen aus anderen Werken als solche kenntlich gemacht habe.
  \\
  \linebreak
  Lemgo, den \today
 \end{sloppypar}
\end{document}
