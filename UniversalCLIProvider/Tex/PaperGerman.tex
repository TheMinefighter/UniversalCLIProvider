\documentclass[a4paper,12pt,titlepage,ngerman]{article}
\usepackage{babel}
\usepackage[T1]{fontenc}
\usepackage[utf8x]{inputenc}
\usepackage[a4paper,margin=2.5cm]{geometry}
\usepackage{csquotes}
%\usepackage{cite}
% Kopf- und Fußzeilen
\usepackage{scrlayer-scrpage}
\setkomafont{pageheadfoot}{\textrm}
\ifoot{Tobias Brohl}
\cfoot{\thepage}
\ohead{}
% Für mathematische Befehle und Symbole
\usepackage{amsmath}
\usepackage{amssymb}

% Für Bilder
\usepackage{graphicx}
\usepackage{float}

% Für Algorithmen
\usepackage{algpseudocode}
\usepackage[backend=biber,style=numeric]{biblatex}
\bibliography{Biblio}
\addbibresource{Biblio.bib}
\usepackage[nottoc,numbib]{tocbibind}
% Für Quelltext
\usepackage{listings}
\usepackage{color}
\definecolor{mygreen}{rgb}{0,0.6,0}
\definecolor{lightgray}{gray}{0.9}
\definecolor{mygray}{rgb}{0.5,0.5,0.5}
\definecolor{mymauve}{rgb}{0.58,0,0.82}
\lstset{
keywordstyle=\color{blue},commentstyle=\color{mygreen},
stringstyle=\color{mymauve},rulecolor=\color{black},
basicstyle=\footnotesize\ttfamily,numberstyle=\tiny\color{mygray},
captionpos=b, % sets the caption-position to bottom
keepspaces=true, % keeps spaces in text
numbers=left, numbersep=5pt,showstringspaces=true,
showtabs=false, stepnumber=2, tabsize=2, title=\lstname,
numbers=none,language={[Sharp]C},breaklines=true,frame=single, literate={ö}{{ö}}1
{ä}{{ä}}1
{ü}{{ü}}1
{ß}{{\ss}}1
}

% Diese beiden Pakete müssen als letztes geladen werden
%\usepackage{hyperref} % Anklickbare Links im Dokument
\usepackage{cleveref}
%\usepackage{biber}

% Daten für die Titelseite
\title{Entwicklung einer Bibliothek zum universellen bereitstellen von Kommandozeilenoberfächen}
\author{Tobias Brohl}
\date{\today \linebreak \linebreak Facharbeit im Fach Informatik (If GK 1)}

\parindent=0pt
\newcommand{\inlinecode}[1]{{\lstinline[]$#1$}}
\begin{document}
 \begin{sloppypar}
  \maketitle
  \setcounter{tocdepth}{5}
  \thispagestyle{empty}
  \tableofcontents
  \pagebreak
  \section{Einleitung \& Motivation}\label{sec:Intro}
   \subsection{Differenzierung verschiedener Typen Kommandozeilenoberflächen}\label{subsec:DifferentCLITypes}
 \subsubsection{Windows typische Kommandozeilenoberflächen}
 Typisch für Windows Kommandozeilenoberflächen ist es ein \inlinecode{/} vor Paramternamen, manchmal auch vor Subkommmando-Namen, verwendet wird, wobei oft auch die Verwendung von \inlinecode{-} korrekt erkannt wird.
 Au\ss erdem werden im Windows Bereich oft sehr viele Funktionen, die z.T. nicht viel miteinander zu tun haben, in ein Befehl gelegt, welches dann viele Subkommandos besitzt.
 \subsubsection{*nix typische Kommandozeilenoberflächen}
 *nix\footnote{*nix wird als Sammelbegriff für Unix und Unix basierte Systeme, wie Linux, MacOS oder OpenBSD, verwendet} typische Kommandozeilenoberflächen
 zeichenen sich dadurch aus das sie statt des \inlinecode{/}, welches zu mehrdeutigkeiten führen könnte, da in *nix systemen Pfade mit Schrägstrichen beginnen, vor Parametern \inlinecode{--} verwenden.
 Häufig benötigte Parameter haben haben meist ein Kurzformen Form eines einzelnen Buchstaben / Zeichen, welcher hinter einem \inlinecode{-} angegeben wird.
 Viele Kommandozeilenoberflächen erlauben auch die aneinander Reihung solcher Kurzformen:
 Wenn z.B. \inlinecode{-a} und \inlinecode{-m} Parameter Kurzformen sind, kann \inlinecode{-am} als Kurzform für beide Optionen verwendet werden.
 Dabei muss \inlinecode{-a} ein Flag\footnote{Als Flag werden Parameter bezeichnet, die keinen weiteren Wert benötigen sondern ein boolean aktivieren} sein,
 \inlinecode{-m} kann auch ein Wertparamter sein dessen Wert dann dahinter stehen müsste.
 
 Eine weitere Eigenschaft, die typisch für Kommandozeilenoberflächen im *nix Berreich ist, ist dass,
 im Vergleich zu Windows Kommandozeilenoberflächen mehr getrennte Befehle mit weniger Subkommandos verwendet werden.
 \subsubsection{Ineraktive Kommandozeilenoberflächen}
 Mit der veröffentlichung von Windows XP Professional im Jahr 2000, wurde die Windows Management Instrumentation (WMI),
 zusammen mit der dafür vorgesehenen Windows Management Instrumentation Command-line (WMIC) eingeführt~\cite{WMIProgrammingBlogPost}.
 In dieser hat das Konzept einer interaktiven Shell zum aufrufen von
 Programmfunktionen~\footnote{Für SSH, TELNET und interaktive Compiler kamen interaktive Kommandozeilenoberflächen auch schon früher zum Einsatz} erstmals zum Einsatz.
 Diese interactive Shell kann mit dem \inlinecode{wmic} Befehl aufgerufen werden, dann kann man Einstellungen definieren wie
 z.B. die Sprache mit \inlinecode{/LOCALE} definieren die bis zum beenden der interaktiven Shell verwendet werden.
 
 Eine weiterer wichtiger Meilenstein für interakive Shells ist die 2005 von Facebook Inc. veröffentlichte Bibliothek Nubia~\cite{NubiaReleaseBlogPost} für Python.
 \begin{figure}[H]
  \includegraphics[width=\linewidth]{PythonNubia.png}
  \caption{Ausschnitt aus der offiziellen Demo der Bibliothek}
  \label{fig:PythonNubia}
 \end{figure}
 Sie hat als erstes Syntax-highlighting in einer ineraktiven Kommandozeile eingeführt, zusammen mit einem Autocomplete auf dem Niveau moderner IDEs.
 \subsubsection{Graphische Kommandozeilenbenutzeroberfächen}
 Graphische Kommandozeilenbenutzeroberfächen (GCLUI) werden oft verwendet wenn die Bedienung durch einen Endnutzer passieren soll,
 und nicht dazu genutzt werden soll um von anderen Programmen automatisch aufgerufen werden soll.
 In vielen solchen Fällen würde man dann Grafische Benutzeroberflächen (GUI) nutzen, in Fällen wo die Bedienbarkeit aber auch über SSH u.ä. gewährleistet sein soll werden aber meist GCLUI genutzt.
 Dass prominenteste Beispiel für einen solchen fall ist der \inlinecode{raspi-config} Befehl der die System-Konfiguration von Raspberry Pi's erlaubt.
 %TODO Plural Raspi
 \begin{figure}[H]
\includegraphics[width=\linewidth]{raspi-config.png}
\caption{Die raspi-config Oberfläche}
\label{fig:raspi-config}
 \end{figure}
 Die Navigation ist dort mit den Pfeil- und Zahlentasten, sowie der Tabtaste, möglich und macht die Bedienung sehr einfach und intuitiv~\cite{RaspiConfigOfficialInfo}.
 \subsection{Vergangenheit \& Aktueller Stand der Technik für C\#}\label{subsec:CurrentState}
 \subsubsection{NDesk.Options / Mono.Options Bibliothek}
 Die NDesk.Options Bibliothek wurde zuerst im Januar 2008 angekündigt~\cite{NDeskAnnouncement},
 und noch im gleichen Monat in erster Version veröffentlicht~\cite{NDesk1stRelease}.
 Der Author hat diese als Nachfolger der Mono.GetOptions Bibliothek,
 welche im Dezember 2006 schon existiert~\cite{MonoGetOptions3rdBlogPost} haben muss und dessen Entwicklung spätestens Ende 2007 eingestellt wurde,
 über die jedoch wenige Informationen vorhanden sind.
 Aufgrund dieses frühen Erscheinens ist die Bibliothek auf C\#2 ausgelegt, weswegen von vielen Möglichkeiten neuerer C\# Versionen,
 die die Bedinung einer solchen Bibliothek deutlich einfacher machen würden viele neue Bedienmöglichkeiten mit sich bringen.
 Die Bibliothek erlaubt das definieren von Optionen optional mit mehrere verwendbaren Namen.
 Diese können entweder ein delegate bereitstellen, dass den folgenden Parameter als String verarbeitet,
 oder ein Typ angegeben in den der folgende Parameter durch einen \inlinecode{System.ComponentModel.TypeConverter} umgewandelt wird,
 und dann ein delegate, dass das Parameter in dem vorher angegebenen Typ verarbeitet.
 Des weiteren werden Boolean flags unterstützt die durch \inlinecode{-OptionsName+} aktiviert
 und durch \inlinecode{-OptiosName-} deaktiviert werden können~\cite{NDeskOptionSetDocumentation}.
 Der letzte eigene Release wurde im Oktober 2008 veröffentlicht~\cite{NDeskOptionsLastRelease},
 dann ist die Bibliothek als Mono.Options als Teil des Mono Projekts neuveröffentlicht worden~\cite{MonoOptions1stCommit},
 dass seit 2011 durch Xamarin~Inc. weiterentwickelt wird~\cite{MonoFutureInterview}, welche seit 2016 Teil von Microsoft ist~\cite{MicrosoftBlogAcquireXamarin}.
 Seit dem ist die Popularität der Bibliothek sehr stark gestiegen, sodass es aktuell kanpp 600.000 Downloads auf nuget.org hat~\cite{MonoOptionsNuget}.
 \subsubsection{Command Line Parser Bibliothek}
 Die 2012 veröffentlichte Bibliothek Command Line Parser~\cite{FirstCommandLineParserCommit}
 \footnote{Der Autor selbst schreibt in der readme des Projekts 2005, jedoch habe ich keine Informationen finden können die dies decken, auf eine Anfrage hat der Autor nicht geantwortet}
 hat als erstes das Konzept von Attribut-definierten Kommandozeilenoberflächen im C\#Bereich.
 Sie wird seit der Veröffentlichung aktiv weiteretwickelt, nutzt die möglichkeiten aktueller C\# Versionen.
 Die Bibliothek stellt eine vollständige Hilfefunktion, baumartige Organistaion von Subkommandos, Parameter, Umfangreiche Typkonvertierungen, sowie Kurzformen bereit~\cite{CommandLineParserWiki}.
 Au\ss erdem stellt die Bibliothek dedizierten F\# Support bereit, der jedoch in dieser Arbeit nicht weiter behandelt werden soll.
 Die Bibliothek ist die am weitesten verbreitete Lösung in diesem Gebiet mit inzwischen über 6.000.000 Downloads auf nuget.org~\cite{CommandLineParserNuget}.

  \section{Bisher ungelöste Probleme}\label{sec:CurrentProblems}
  \subsection{Sonderzeichen-Unterstützung}
  \subsection{Interaktive Konsolen}
  \subsection{Konfigurationsdateimanagement}
  \section{Implementation}\label{sec:Content}
  \subsection{Aufbau der Implementation}\label{subsec:Architecture}
  \subsubsection{UniversalCLIProvider.Attributes}
  \subsubsection{UniversalCLIProvider.Interpreters}
  \subsubsection{UniversalCLIProvider.Internals}
  \subsection{Erläuterung der Implementation der Klasse ContextInterpreter}\label{subsec:ConextInterpreter}
  \subsection{Demonstration der Funktionen anhand der Referenzverwendung}\label{subsec:demonstration}
  \subsection{Vergleich meiner Lösung mit bisherigen Lösungen}\label{subsec:Comparison}
  \section{Ausblick}\label{sec:Future}
  \subsection{Ausbau des Funktionsumfangs}\label{subsec:MoreFunctions}
  \subsection{Übertragbarkeit auf andere Programmiersprachen}\label{subsec:PortabilityToOtherLangs}
  \subsection{Geschwindigkeitsverbesserungen durch Speicherung von Reflections}\label{subsec:StoringReflections}
  \subsection{Implementation einer eigenen Auto-Vervollständigung für Linux}\label{subsec:Autocomplete}
  \subsection{Verbesserung der Unit test Abdeckung}\label{subsec:MoreUnitTests}
  \section{Anmerkungen}\label{sec:AdditionalNotes}
  \subsection{Umgebungsvorraussetzuungen zum nutzen der Bibliothek}\label{subsec:SystemRequirements}
  \subsection{Verwendung von weiteren Bibliotheken}\label{subsec:UsageOfLibraries}
  \section{Fazit}\label{sec:Conclusion}
  \section{Anhang}\label{sec:anhang}
  Alle Anhänge werden in digitaler Form auf CD bereitgestellt.
  \section{Literaturverzeichnis}\label{sec:Literature}
  \printbibliography[heading=none]
  \section{Selbstständigkeitserklärung}\label{sec:IDidThisMyself}
  Hiermit erkläre ich, dass ich die vorliegende Facharbeit selbstständig und ohne fremde Hilfe angefertigt und nur die im Literaturverzeichnis
  aufgeführten Hilfen und Quellen benutzt habe.
  Insbesondere versichere ich, dass ich alle wörtlichen
  und sinngemä\ss en Übernahmen aus anderen Werken als solche kenntlich gemacht habe.
  \medskip
  Lemgo, den \today
  \medskip
 \end{sloppypar}
\end{document}
